%%%%%%%%%%%%%%%%%%%%%%%%%%%%%%%%%%%%%%%%%%%%%%%%%%%%%%%%%%%%%%%%%%%%%%%%%
%%
%W  aclib.tex                                               Karel Dekimpe
%W                                                           Bettina Eick
%%

%%%%%%%%%%%%%%%%%%%%%%%%%%%%%%%%%%%%%%%%%%%%%%%%%%%%%%%%%%%%%%%%%%%%%%%%%%%%%
\Chapter{The Almost Crystallographic Groups Package}

A group is called *almost crystallographic* if it is a finitely generated 
nilpotent-by-finite group without non-trivial finite normal subgroups. An 
important special case of almost crystallographic groups are the *almost 
Bieberbach groups*: these are almost crystallographic and torsion free. 

By its definition, an almost crystallographic group $G$ has a finitely 
generated nilpotent normal subgroup $N$ of finite index. Clearly, $N$ is 
polycyclic and thus has a polycyclic series. The number of infinite cyclic 
factors in such a series for $N$ is an invariant of $G$: the *Hirsch length* 
of $G$. 

For each almost crystallographic group of Hirsch length 3 and 4 there exists 
a representation as a rational matrix group in dimension 4 or 5, respectively. 
These representations can be considered as affine representations of dimension
3 or 4. Via these representations, the almost crystallographic groups act 
(properly discontinuously) on $\R^3$ or $\R^4$. That is one reason to define
the *dimension* of an almost crystallographic group as its Hirsch length.

The 3-dimensional and a part of the 4-dimensional almost crystallographic 
groups have been classified by K. Dekimpe in \cite{KD}. This classification 
includes all almost Bieberbach groups in dimension 3 and 4. It is the first
central aim of this package to give access to the resulting library of groups.
The groups in this electronic catalog are available in two different 
representations: as rational matrix groups and as polycyclically presented
groups. While the first representation is the more natural one, the latter
description facilitates effective computations with the considered groups
using the methods of the {\sf Polycyclic} package.

The second aim of this package is to introduce a variety of algorithms for
computations with polycyclically presented almost crystallographic groups. 
These algorithms supplement the methods available in the {\sf Polycyclic} 
package and give access to some methods which are interesting specifically 
for almost crystallographic groups. In particular, we present methods to
compute Betti numbers and to construct or check the existence of certain 
extensions of almost crystallographic groups. We note that these methods 
have been applied in \cite{DE1} and \cite{DE2} for computations with 
almost crystallographic groups.

Finally, we remark that almost crystallographic groups can be seen as natural 
generalizations of crystallographic groups. A library of crystallographic 
groups and algorithms to compute with crystallographic groups are available 
in the \GAP\ packages `cryst', `carat' and `crystcat'. 

%%%%%%%%%%%%%%%%%%%%%%%%%%%%%%%%%%%%%%%%%%%%%%%%%%%%%%%%%%%%%%%%%%%%%%%%%%%%%
\Section{More about almost crystallographic groups}

Almost crystallographic groups were first discussed in the theory of 
actions on Lie groups.  We recall the original definition here briefly
and we refer to \cite{AUS}, \cite{KD} and \cite{LEE} for more details. 

Let $L$ be a connected and simply connected nilpotent Lie group. For 
example, the 3-dimensional Heisenberg group, consisting of all upper 
unitriangular $3\times3$--matrices with real entries is of this type.
Then $L\rtimes {\rm Aut}(L)$ acts affinely (on the left) on $L$ via
$$ \forall l,l'\in L,\forall \alpha \in {\rm Aut}(L):\; 
   ^{(l,\alpha)}l'=l \, \alpha(l').  $$

Let $C$ be a maximal compact subgroup of ${\rm Aut}(L)$. Then a subgroup $G$ 
of $L \rtimes C$ is said to be an almost crystallographic group if and only 
if the action of $G$ on $L$, induced by the action of $L\rtimes {\rm Aut}(L)$,
is properly discontinuous and the quotient space $G \backslash L$ is compact. 
One recovers the situation of the ordinary crystallographic groups by taking 
$L={\Bbb R}^n$, for some $n$, and $C=O(n)$, the orthogonal group.

More generally, we say that an abstract group is an almost crystallographic
group if it can be realized as a genuine almost crystallographic subgroup 
of some $L \rtimes C$. In the following theorem we outline some algebraic 
characterizations of almost crystallographic groups; see Theorem 3.1.3 of 
\cite{KD}. Recall that the *Fitting subgroup Fitt$(G)$* of a 
polycyclic-by-finite group $G$ is its unique maximal normal nilpotent 
subgroup.

\proclaim Theorem. 
The following are equivalent for a polycyclic-by-finite group $G$:
\parindent  30pt
\item{(1)} $G$ is an almost crystallographic group.
\item{(2)} Fitt$(G)$ is torsion free and of finite index in $G$.
\item{(3)} $G$ contains a torsion free nilpotent normal subgroup $N$
of finite index in $G$ with $C_G(N)$ torsion free.
\item{(4)} $G$ has a nilpotent subgroup of finite index and there
are no non-trivial finite normal subgroups in $G$.
\medskip
\parindent 0pt

In particular, if $G$ is almost crystallographic, then $G / Fitt(G)$
is finite. This factor is called the *holonomy group* of $G$. 

The dimension of an almost crystallographic group equals the dimension
of the Lie group $L$ above which coincides also with the Hirsch length 
of the polycyclic-by-finite group. This library therefore contains 
families of virtually nilpotent groups of Hirsch length 3 and 4. 

